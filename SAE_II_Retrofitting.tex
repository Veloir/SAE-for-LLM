\documentclass[11pt]{article}

% ===== Encoding & Fonts =====
\usepackage[T1]{fontenc}
\usepackage[utf8]{inputenc}
\usepackage{lmodern}
\usepackage{microtype}

% ===== Page Layout =====
\usepackage[a4paper,margin=1in]{geometry}
\linespread{1.06}
\setlength{\parskip}{0.4em}
\setlength{\parindent}{12pt}

% ===== Math & Theorems =====
\usepackage{amsmath,amssymb,amsthm,mathtools,bm}
\theoremstyle{plain}
\newtheorem{theorem}{Theorem}[section]
\newtheorem{lemma}[theorem]{Lemma}
\newtheorem{corollary}[theorem]{Corollary}
\theoremstyle{definition}
\newtheorem{definition}[theorem]{Definition}
\theoremstyle{remark}
\newtheorem{remark}[theorem]{Remark}
\newtheorem{proposition}{Proposition}

% ===== Algorithms =====
\usepackage{algorithm}
\usepackage{algorithmic} % 若更喜欢 algpseudocode 语法,可换为 \usepackage{algpseudocode}

% ===== Figures / Tables =====
\usepackage{graphicx}
\usepackage{booktabs}
\usepackage{enumitem}

% ===== References =====
\usepackage[numbers,sort&compress]{natbib}

% ===== Hyperlinks =====
\usepackage{hyperref}
\hypersetup{
  colorlinks=true,
  linkcolor=black,
  citecolor=black,
  urlcolor=blue
}

% ===== Header & Footer =====
\usepackage{fancyhdr}
\fancypagestyle{firstpage}{%
  \fancyhf{}
\fancyfoot[C]{Email: \texttt{veloir@163.com} \quad ORCID: \href{https://orcid.org/0009-0004-4194-5999}{0009-0004-4194-5999}}
  \renewcommand{\headrulewidth}{0pt}
  \renewcommand{\footrulewidth}{0pt}
}

% ===== Custom Commands (safe in/out of math mode) =====
\newcommand{\Exist}{\ensuremath{\mathsf{Exist}}}
\newcommand{\Discern}{\ensuremath{\mathsf{Discern}}}
\newcommand{\Free}{\ensuremath{\mathsf{Free}}}
\newcommand{\Synchrony}{\ensuremath{\mathsf{Synchrony}}}

% ===== Title & Author =====
\title{\vspace{-0.5em}\textbf{Retrofitting Large Language Models with Logos-SAE}\vspace{-0.3em}}
\author{Hui Xu}
\date{\small\today}

\begin{document}
\maketitle
\thispagestyle{firstpage}

\begin{abstract}
This paper is a companion to our previous work \emph{Structural Axiom of Existence: Logos for Aware LLMs}, which focused on constructing new models under the SAE framework. Here we address the complementary task of \emph{retrofitting existing LLMs}. The \textbf{Structural Axiom of Existence (SAE)} originates from the foundation
\[
\Exist(X) := \Discern(X) \wedge \Free(X),
\]
where \Discern{} denotes a boundary-making capacity ensuring recognizability and operability, and \Free{} denotes an openness enabling generativity and transformation. When \Discern{} and \Free{} act in sustained coordination, their trajectory gives rise to \Synchrony{}. Hence the working formulation of SAE for system design is
\[
\text{SAE} = \Discern \wedge \Free \wedge \Synchrony.
\]
Guided by this principle, we show how existing LLM architectures can be retrofitted to reduce hallucinations, improve energy efficiency, and enhance interpretability, without sacrificing generative capability.
\end{abstract}

\section{Introduction}

Large Language Models (LLMs) have become a cornerstone of natural language processing and artificial intelligence. Yet they continue to face several fundamental issues:
\begin{itemize}
  \item \textbf{Hallucination:} generation of false or inconsistent content;
  \item \textbf{High energy cost:} redundant or invalid reasoning paths waste computation;
  \item \textbf{Low interpretability:} opaque mechanisms hinder verification and control;
  \item \textbf{Heavy reliance on external alignment:} post-hoc fixes such as RLHF lack intrinsic guarantees.
\end{itemize}

We propose SAE as a unifying framework for \emph{retrofitting existing LLMs}.  
Rooted in its structural foundation (\(\Exist = \Discern \wedge \Free\)) and extended through the principle of \Synchrony{}, SAE provides a systematic way to embed structural constraints in Transformer models. By integrating \Discern{} with \Free{} at every stage, and enforcing their \Synchrony{} across input, representation, and decoding, we obtain \textbf{Logos-SAE LLMs} that are more reliable, efficient, and interpretable.

\section{SAE Foundation}

\subsection{Structural Axiom of Existence}

The foundational axiom states:
\[
\Exist(X) := \Discern(X) \wedge \Free(X).
\]

A phenomenon $X$ exists structurally iff it simultaneously exhibits:
\begin{itemize}
  \item \Discern: boundary-making capacity granting recognizability and operability;
  \item \Free: openness granting generativity and transformation.
\end{itemize}

When these two aspects maintain coherence along generative trajectories, we obtain \Synchrony{}.  
Thus, the working formulation for LLM design becomes:
\[
\text{SAE} = \Discern \wedge \Free \wedge \Synchrony.
\]


\subsection{Problems of Current LLMs}
\begin{itemize}
  \item \textbf{Strong Free, weak Discern:} Attention and Softmax ensure probabilistic generation (Free), while Discern only enters indirectly via external alignment (RLHF).
  \item \textbf{Lack of Synchrony:} many tokens are generated without structural validity, causing hallucinations and wasted computation.
\end{itemize}

\subsection{SAE Improvement Strategy}
\begin{itemize}
  \item Embed Discern into each layer: input, attention, softmax;
  \item Add Synchrony constraints into training objectives;
  \item Perform real-time filtering during decoding, acting as an ``intrinsic overseer.''
\end{itemize}


\section{SAE-LLM Architecture (Technical Details)}

\subsection{Input Layer: Discern-aware Embedding}
Extend the embedding to include a \textbf{Discern vector $D$}:
\[
E = [E^{\text{token}}, E^{\text{pos}}, E^{\text{discern}}],
\]
where:
\begin{itemize}
  \item $E^{\text{token}}$ = token embedding;
  \item $E^{\text{pos}}$ = positional embedding;
  \item $E^{\text{discern}}$ = discernment embedding (derived from syntax checks, consistency verification, knowledge retrieval, or energy preference).
\end{itemize}

\begin{verbatim}
def embed(token, position, context):
    token_vec = token_embedding[token]
    pos_vec = pos_embedding[position]
    discern_vec = compute_discern(token, context)
    return concat([token_vec, pos_vec, discern_vec])
\end{verbatim}

\subsection{Representation Layer: SAE-Attention}
Standard attention:
\[
\alpha_{ij} = \frac{\exp(Q_i K_j^\top / \sqrt{d})}{\sum_k \exp(Q_i K_k^\top / \sqrt{d})}.
\]
SAE modification:
\[
\alpha_{ij} = \frac{\exp(Q_i K_j^\top / \sqrt{d}) \cdot D_j}{\sum_k \exp(Q_i K_k^\top / \sqrt{d}) \cdot D_k}.
\]

\begin{verbatim}
def sae_attention(Q, K, V, D):
    scores = (Q @ K.T) / sqrt(d)
    scores = scores + log(D + eps)   # Discern correction
    attn = softmax(scores, dim=-1)
    return attn @ V
\end{verbatim}

\subsection{Output Layer: SAE-Softmax}
Standard softmax:
\[
p_i = \frac{\exp(z_i)}{\sum_j \exp(z_j)}.
\]
SAE modification:
\[
p_i = \frac{\exp(z_i) \cdot D_i}{\sum_j \exp(z_j) \cdot D_j}.
\]

\begin{verbatim}
def sae_softmax(logits, D, dim=-1, eps=1e-12):
    adj_logits = logits + torch.log(D + eps)
    return F.softmax(adj_logits, dim=dim)
\end{verbatim}

\subsection{Training Objective: Synchrony Optimization}
Standard cross-entropy loss:
\[
\mathcal{L}_{\text{task}} = -\sum_t \log p(y_t).
\]
Synchrony score:
\[
\mathcal{E} = \frac{\sum_i p_i D_i}{\sum_i p_i}.
\]
Training loss:
\[
\mathcal{L} = \mathcal{L}_{\text{task}} + \alpha (1-\mathcal{E}) + \beta \mathcal{L}_{\text{energy}}.
\]

\subsection{Decoding: SAE-Decoding}
Traditional decoding: sample from $p_i$. SAE modification: filter candidates with low synchrony.
\begin{verbatim}
def sae_decoding_step(logits, D):
    probs = sae_softmax(logits, D)
    sync_probs = probs * D
    next_token = torch.argmax(sync_probs)
    return next_token, sync_probs
\end{verbatim}

\section{System View: Dual-Stream Mechanism}

SAE-LLM is a \textbf{dual-stream system}:
\begin{itemize}
  \item \textbf{Free stream:} explores generative possibilities (probability distribution);
  \item \textbf{Discern stream:} scores each candidate token by structural validity;
  \item \textbf{Synchrony module:} integrates the two streams for the final decision.
\end{itemize}

Formal expression:
\[
\text{Output}(t) = \arg\max_{v \in \mathcal{V}} \; p(v) \cdot D(v).
\]

\begin{verbatim}
def sae_step(Q, K, V, logits, context):
    attn_out = sae_attention(Q, K, V, D=None)  # Free stream
    D = compute_discern_logits(logits, context) # Discern stream
    next_token, probs = sae_decoding_step(logits, D)
    return next_token, probs
\end{verbatim}

\section{Expected Benefits}
\begin{itemize}
  \item Lower hallucination rate: invalid tokens ($D=0$) are filtered out;
  \item Energy efficiency: low-discernment paths are pruned early;
  \item Better interpretability: Discern sources are trackable, making decisions explainable.
\end{itemize}

\section{Conclusion}
SAE-LLM introduces a new paradigm:
\begin{itemize}
  \item Embeds Discern within every LLM layer;
  \item Achieves generation-validation synchrony;
  \item Provides an executable framework: formulas, pseudocode, and implementation pathway.
\end{itemize}

% =================== Module Algorithms ===================

\section{SAE-Attention (Algorithm)}

\subsection{Formula}
\[
\alpha_{ij} = \frac{\exp(Q_i K_j^\top / \sqrt{d}) \cdot D_j}{\sum_k \exp(Q_i K_k^\top / \sqrt{d}) \cdot D_k}, 
\qquad
\text{SAE-Attn}(Q,K,V)_i = \sum_j \alpha_{ij} V_j.
\]

\subsection{Algorithm}

\begin{algorithm}[h]
\caption{SAE-Attention}
\begin{algorithmic}[1]
\REQUIRE Queries $Q$, Keys $K$, Values $V$, Discern weights $D$
\STATE Compute similarity scores: $S = QK^\top / \sqrt{d}$
\STATE Adjust scores with Discern: $S \gets S + \log(D + \varepsilon)$
\STATE Apply softmax: $\alpha = \text{softmax}(S)$
\STATE Weighted aggregation: $O = \alpha V$
\RETURN $O, \alpha$
\end{algorithmic}
\end{algorithm}

\section{SAE-Softmax (Algorithm)}

\subsection{Formula}
\[
p_i = \frac{\exp(z_i)\cdot D_i}{\sum_j \exp(z_j)\cdot D_j}.
\]

\subsection{Algorithm}

\begin{algorithm}[h]
\caption{SAE-Softmax}
\begin{algorithmic}[1]
\REQUIRE Logits $z$, Discern weights $D$
\STATE Adjusted logits: $z' \gets z + \log(D + \varepsilon)$
\STATE Compute probabilities: $p = \text{softmax}(z')$
\RETURN $p$
\end{algorithmic}
\end{algorithm}

\section{Synchronization-Aware Training (Algorithm)}

\subsection{Formula}
Task loss:
\[
\mathcal{L}_{\text{task}} = -\sum_t \log p(y_t).
\]
Synchrony:
\[
\mathcal{E} = \frac{\sum_i p_i D_i}{\sum_i p_i}.
\]
Overall loss:
\[
\mathcal{L} = \mathcal{L}_{\text{task}} + \alpha (1-\mathcal{E}) + \beta \mathcal{L}_{\text{energy}}.
\]

\subsection{Algorithm}

\begin{algorithm}[h]
\caption{SAE-Synchrony Training}
\begin{algorithmic}[1]
\REQUIRE Model parameters $\theta$, Discern estimator $\phi$
\FOR{each training step}
  \STATE Forward pass: compute logits $z$ and probs $p$
  \STATE Compute Discern weights $D = D_\phi(x)$
  \STATE Compute synchrony $\mathcal{E} = \sum_i p_i D_i / \sum_i p_i$
  \STATE Compute loss $\mathcal{L} = \mathcal{L}_{task} + \alpha(1-\mathcal{E}) + \beta\mathcal{L}_{energy}$
  \STATE Backpropagation: update $\theta, \phi$
\ENDFOR
\end{algorithmic}
\end{algorithm}

\section{SAE-Decoding (Algorithm)}

\subsection{Formula}
\[
\text{Output}(t) = \arg\max_{v \in \mathcal{V}} p(v) \cdot D(v).
\]

\subsection{Algorithm}

\begin{algorithm}[h]
\caption{SAE-Decoding}
\begin{algorithmic}[1]
\REQUIRE Logits $z$, Discern weights $D$
\STATE Compute SAE-Softmax: $p = \text{sae-softmax}(z,D)$
\STATE Synchronize: $p' \gets p \cdot D$
\STATE Select token: $t^* = \arg\max p'$
\RETURN $t^*, p'$
\end{algorithmic}
\end{algorithm}

\section{Full SAE-LLM Pipeline}

The complete execution flow of SAE-LLM from input to output:

\begin{algorithm}[h]
\caption{SAE-LLM Pipeline}
\begin{algorithmic}[1]
\REQUIRE Input sequence $x_{1:T}$
\STATE \textbf{Embedding:} tokenize, embed, and compute initial Discern weights
\STATE \textbf{Transformer layers:}
  \FOR{each layer $l=1\ldots L$}
    \STATE Compute SAE-Attention with Discern
    \STATE Apply FFN + residuals
    \STATE Update Discern estimator $D^{(l)}$
  \ENDFOR
\STATE \textbf{Output:} SAE-Softmax with $D^{(L)}$
\STATE \textbf{Synchrony metric:} $\mathcal{E} = \sum_i p_i D_i / \sum_i p_i$
\STATE \textbf{Training:} minimize $\mathcal{L} = \mathcal{L}_{task} + \alpha (1-\mathcal{E}) + \beta \mathcal{L}_{energy}$
\STATE \textbf{Decoding:} select tokens ensuring $\Free \wedge \Discern$
\RETURN Output sequence $\hat{y}_{1:T}$
\end{algorithmic}
\end{algorithm}

\section{Implementation Notes (Minimal PyTorch Demo)}

This section summarizes practical choices and provides a minimal, end-to-end PyTorch sketch to reproduce SAE-attention, SAE-softmax, the synchrony-aware loss, and decoding. The code is intentionally lightweight and omits data/loader details.

\subsection{Design Choices and Tips}
\begin{itemize}
  \item \textbf{Discern estimator $D_\phi$:} start with a simple MLP head on top of contextual hidden states; optionally fuse external signals (syntax checks, NLI scores, retrieval hits). Use \texttt{sigmoid} to bound $D\in[0,1]$.
  \item \textbf{Gradient flow:} when $D$ comes from hard constraints (e.g., grammar mask), detach gradients (no backprop through $D$). For learned $D_\phi$, allow gradients but consider a smaller LR and EMA to stabilize.
  \item \textbf{Stability:} always add \texttt{eps} and use \texttt{log(D + eps)} in logits or attention scores. Clip or floor $D$ to avoid $\log 0$.
  \item \textbf{Energy proxy:} define \texttt{cost} per token step (e.g., retrieval depth, KV-cache growth, tool-call flag) and penalize via $\mathcal{L}_{\text{energy}}$.
  \item \textbf{Curriculum:} ramp \(\alpha\) (sync weight) from 0 to target over first epochs; similarly ramp macro-scale constraints if any.
  \item \textbf{Evaluation:} report hallucination rate (e.g., TruthfulQA), task accuracy, \emph{Joules/token} (or FLOPs/token), and the synchrony score \(\mathcal{E}=\sum p_i D_i / \sum p_i\).
\end{itemize}

\subsection{Minimal Modules}

\begin{verbatim}
import torch
import torch.nn as nn
import torch.nn.functional as F

class SAEAttention(nn.Module):
    def __init__(self, d_model, n_heads):
        super().__init__()
        assert d_model % n_heads == 0
        self.d_model = d_model
        self.n_heads = n_heads
        self.d_head = d_model // n_heads
        self.Wq = nn.Linear(d_model, d_model)
        self.Wk = nn.Linear(d_model, d_model)
        self.Wv = nn.Linear(d_model, d_model)
        self.Wo = nn.Linear(d_model, d_model)

    def forward(self, x, discern=None, attn_mask=None):
        """
        x: (B, T, d_model)
        discern: (B, T) in [0,1] for keys (and optionally values)
        """
        B, T, D = x.size()
        q = self.Wq(x).view(B, T, self.n_heads, self.d_head).transpose(1, 2)  # (B,H,T,dh)
        k = self.Wk(x).view(B, T, self.n_heads, self.d_head).transpose(1, 2)  # (B,H,T,dh)
        v = self.Wv(x).view(B, T, self.n_heads, self.d_head).transpose(1, 2)  # (B,H,T,dh)
        scores = torch.matmul(q, k.transpose(-2, -1)) / (self.d_head ** 0.5)   # (B,H,T,T)
        if attn_mask is not None:
            scores = scores.masked_fill(attn_mask, float('-inf'))
        if discern is not None:
            # expand discern (B,T)->(B,1,1,T) and add log(D+eps)
            Dk = torch.clamp(discern, min=1e-12).unsqueeze(1).unsqueeze(1)
            scores = scores + torch.log(Dk)
        attn = torch.softmax(scores, dim=-1)
        out = torch.matmul(attn, v)                                           # (B,H,T,dh)
        out = out.transpose(1, 2).contiguous().view(B, T, D)                  # (B,T,D)
        return self.Wo(out), attn

class DiscernHead(nn.Module):
    def __init__(self, d_model, hidden=256):
        super().__init__()
        self.net = nn.Sequential(
            nn.Linear(d_model, hidden),
            nn.ReLU(),
            nn.Linear(hidden, 1)
        )
    def forward(self, h):
        # h: (B,T,D) -> D in [0,1]
        return torch.sigmoid(self.net(h)).squeeze(-1)

def sae_softmax(logits, D, dim=-1, eps=1e-12):
    # logits: (B,V), D: (B,V) in [0,1]
    D_safe = torch.clamp(D, min=eps)
    return F.softmax(logits + torch.log(D_safe), dim=dim)

class SAEBlock(nn.Module):
    def __init__(self, d_model, n_heads, d_ff):
        super().__init__()
        self.attn = SAEAttention(d_model, n_heads)
        self.ln1 = nn.LayerNorm(d_model)
        self.ff = nn.Sequential(
            nn.Linear(d_model, d_ff), nn.GELU(), nn.Linear(d_ff, d_model)
        )
        self.ln2 = nn.LayerNorm(d_model)
        self.discern_head = DiscernHead(d_model)

    def forward(self, x, attn_mask=None, discern_keys=None):
        # Discern for keys can be computed from previous hidden states
        attn_out, _ = self.attn(x, discern=discern_keys, attn_mask=attn_mask)
        x = self.ln1(x + attn_out)
        ff_out = self.ff(x)
        x = self.ln2(x + ff_out)
        # produce updated discern scores from current states
        D = self.discern_head(x)  # (B,T)
        return x, D

class SAETransformerLM(nn.Module):
    def __init__(self, vocab_size, d_model=512, n_heads=8, d_ff=2048, n_layers=6):
        super().__init__()
        self.embed = nn.Embedding(vocab_size, d_model)
        self.pos = nn.Embedding(4096, d_model)
        self.blocks = nn.ModuleList([SAEBlock(d_model, n_heads, d_ff) for _ in range(n_layers)])
        self.ln_f = nn.LayerNorm(d_model)
        self.lm_head = nn.Linear(d_model, vocab_size, bias=False)

    def forward(self, idx, attn_mask=None, external_D_vocab=None):
        """
        idx: (B,T) token ids
        external_D_vocab: optional (B,T,V) discern over vocab at each step
        """
        B, T = idx.size()
        pos = torch.arange(T, device=idx.device).unsqueeze(0).expand(B, T)
        h = self.embed(idx) + self.pos(pos)              # (B,T,D)
        D_keys = None
        for blk in self.blocks:
            h, D_keys = blk(h, attn_mask=attn_mask, discern_keys=D_keys)
        h = self.ln_f(h)
        logits = self.lm_head(h)                         # (B,T,V)
        # compute vocab-level discern; default from last hidden state
        if external_D_vocab is not None:
            D_vocab = external_D_vocab
        else:
            # simple projection: reuse lm_head weights for a score, then sigmoid
            D_vocab = torch.sigmoid(logits.detach())     # detach if you want fixed D
        return logits, D_keys, D_vocab
\end{verbatim}

\subsection{Synchrony Loss and Training Loop}

\begin{verbatim}
def synchrony_score(probs, D_vocab, eps=1e-12):
    # both shape (B,T,V)
    num = (probs * D_vocab).sum(dim=-1)          # (B,T)
    den = probs.sum(dim=-1).clamp_min(eps)       # (B,T)
    return (num / den).mean()                    # scalar

def sae_loss(logits, targets, D_vocab, alpha=0.1, beta=0.0, cost=None):
    # logits: (B,T,V), targets: (B,T), D_vocab: (B,T,V)
    probs = F.softmax(logits, dim=-1)
    ce = F.cross_entropy(logits.view(-1, logits.size(-1)),
                         targets.view(-1), reduction='mean')
    E = synchrony_score(probs, D_vocab)
    L_sync = 1.0 - E
    if cost is not None:
        # cost: (B,T,V) or (B,T) proxy
        if cost.dim() == 3:
            L_energy = (probs * cost).sum(dim=-1).mean()
        else:
            L_energy = cost.mean()
    else:
        L_energy = torch.tensor(0.0, device=logits.device)
    return ce + alpha * L_sync + beta * L_energy, dict(ce=ce, E=E, L_sync=L_sync,
                                                       L_energy=L_energy)

# ---- training step (sketch) ----
def train_step(model, batch, optimizer, alpha=0.1, beta=0.0):
    model.train()
    idx, targets = batch['input_ids'], batch['labels']
    logits, D_keys, D_vocab = model(idx)  # external_D_vocab=None
    loss, logs = sae_loss(logits, targets, D_vocab, alpha=alpha, beta=beta)
    optimizer.zero_grad()
    loss.backward()
    torch.nn.utils.clip_grad_norm_(model.parameters(), 1.0)
    optimizer.step()
    return {k: v.item() for k, v in logs.items()} | {'loss': loss.item()}
\end{verbatim}

\subsection{SAE-Decoding (Greedy)}

\begin{verbatim}
@torch.no_grad()
def sae_greedy_decode(model, start_ids, max_len=64, temperature=1.0, eps=1e-12):
    model.eval()
    idx = start_ids
    for _ in range(max_len):
        logits, D_keys, D_vocab = model(idx)
        logits_step = logits[:, -1, :] / max(temperature, 1e-6)  # (B,V)
        D_step = D_vocab[:, -1, :].clamp_min(eps)                # (B,V)
        probs = F.softmax(logits_step + torch.log(D_step), dim=-1)
        # synchrony-weighted greedy choice
        sync_probs = probs * D_step
        next_token = torch.argmax(sync_probs, dim=-1, keepdim=True)  # (B,1)
        idx = torch.cat([idx, next_token], dim=1)
    return idx
\end{verbatim}

\subsection{Hooking External Discern Signals}

\begin{itemize}
  \item \textbf{Hard masks} (grammar, JSON schema): set $D=0$ for disallowed tokens.
  \item \textbf{Retrieval consistency:} boost $D$ for tokens supported by retrieved passages.
  \item \textbf{NLI verifier:} compute entailment scores and map to $D\in[0,1]$.
  \item \textbf{Energy-aware prior:} $D = \exp(-\lambda \cdot \mathrm{cost})$ for expensive operations.
\end{itemize}

\subsection{Recommended Hyperparameters (Starting Points)}
\begin{itemize}
  \item \(\alpha\) (synchrony weight): 0.05--0.2; warm up from 0 over 5--10\% of steps.
  \item \(\beta\) (energy weight): 0.0 initially; introduce 0.01--0.1 after convergence.
  \item Discern head LR: 2--3$\times$ main LR; consider EMA(0.95) for $D$ stabilization.
  \item Floor $D$: use \texttt{clamp\_min(1e-6)}, and consider temperature tuning at decode.
\end{itemize}

\subsection{Repro Checklist}
\begin{itemize}
  \item Log \(\mathcal{E}\) per step and per layer; correlate with hallucination errors.
  \item Ablate: vanilla vs.\ SAE-Softmax only vs.\ SAE-Attention only vs.\ full SAE.
  \item Report Joules/token (or FLOPs/token) and latency alongside task metrics.
  \item Verify that $D$ improves with training (calibration plots, AUC vs.\ weak labels).
\end{itemize}

\newpage
\section{Expected Changes under SAE-Based Redesign}

From the perspective of the \emph{Structural Axiom of Existence} (SAE $=$ Discern $\wedge$ Free), redesigning LLMs with SAE-Attention, SAE-Softmax, and synchronization-aware training is expected to yield the following changes:

\subsection{Generation Quality}
\begin{itemize}
  \item \textbf{Reduced hallucinations:} tokens or trajectories with $D=0$ are pruned in both attention and softmax, suppressing structurally invalid generations.
  \item \textbf{Higher structural consistency:} outputs such as JSON, code, or proofs will adhere more strongly to structural rules, with synchrony $\mathcal{E}$ pushing probability mass toward valid candidates.
  \item \textbf{Creativity preserved:} the \emph{Free} component (distributional diversity) remains intact, so generative flexibility is not lost.
\end{itemize}

\subsection{Training and Convergence}
\begin{itemize}
  \item \textbf{Faster and more stable convergence:} synchronization regularizers act as intrinsic rewards, pruning meaningless updates during training.
  \item \textbf{Lower risk of overfitting:} SAE requires alignment across structural factors, not just memorization of token frequencies.
  \item \textbf{Multi-scale consistency:} discern signals can propagate across token, span, and document levels, ensuring structural synchrony across scales.
\end{itemize}

\subsection{Inference Efficiency and Energy}
\begin{itemize}
  \item \textbf{Energy reduction:} paths with low discernment are excluded early, reducing FLOPs/token and KV-cache growth.
  \item \textbf{Lower latency:} fewer candidate expansions enable faster decoding and convergence in beam search or sampling.
\end{itemize}

\subsection{Interpretability and Controllability}
\begin{itemize}
  \item \textbf{More interpretable attention maps:} attention reflects both similarity and structural validity, clarifying which tokens are structurally acceptable.
  \item \textbf{Enhanced controllability:} external rules (syntax, domain knowledge, safety constraints) can be encoded in $D$, directly influencing behavior without relying solely on RLHF.
\end{itemize}

\subsection{Risks and Trade-offs}
\begin{itemize}
  \item \textbf{Potential reduction in creativity:} overly strict discernment may make outputs more conservative, limiting boundary-pushing innovation.
  \item \textbf{Critical dependence on discern quality:} if the estimator $D_\phi$ is weak, it may over-filter or under-filter, harming usability.
  \item \textbf{Increased training complexity:} additional heads, regularizers, and cost proxies introduce more hyperparameters and tuning overhead.
\end{itemize}

\subsection{Summary}
An SAE-based LLM is expected to become \emph{more reliable, consistent, energy-efficient, and controllable}, while retaining creativity. The main trade-off lies in balancing Discern strictness with generative freedom.

\subsection{Before vs.\ After SAE Redesign}

\begin{table}[h]
\centering
\renewcommand{\arraystretch}{1.3}
\begin{tabular}{p{3.5cm} p{5.5cm} p{5.5cm}}
\toprule
\textbf{Aspect} & \textbf{Standard LLMs (Before SAE)} & \textbf{SAE-Based LLMs (After Redesign)} \\
\midrule
\textbf{Generation Quality} 
& High freedom, but frequent hallucinations; weak structural guarantees 
& Reduced hallucinations via $D$-weights; stronger structural consistency with $\mathsf{Discern} \wedge \mathsf{Free}$ \\

\textbf{Training and Convergence} 
& Cross-entropy only; slow convergence; prone to memorization 
& Synchronization regularizers; faster convergence; lower overfitting; multi-scale consistency \\

\textbf{Inference Efficiency} 
& High energy use; redundant token expansions; large KV-cache growth 
& Energy-efficient pruning; only discern-valid paths proceed; reduced FLOPs/token \\

\textbf{Interpretability} 
& Attention reflects only similarity; limited explainability 
& Attention integrates discernment; structurally meaningful maps \\

\textbf{Controllability} 
& RLHF as external patch; fragile, expensive to align 
& $D$ integrates domain/safety rules directly; controllable at inference time \\

\textbf{Risks / Trade-offs} 
& Creativity unbounded but unreliable 
& Reliability improved, but creativity may shrink if $D$ overly strict; requires high-quality discern estimator \\
\bottomrule
\end{tabular}
\caption{Comparison of standard Transformer-based LLMs and SAE-based redesigned LLMs.}
\end{table}

\newpage
\section{Systematic SAE Redesign of LLM Components}

In addition to modifying attention, softmax, and training objectives, the
Structural Axiom of Existence (SAE: $\mathsf{Exist}(X) = \mathsf{Discern}(X) \wedge \mathsf{Free}(X)$)
suggests a systematic redesign of almost every core mechanism in LLMs.
We outline below seven additional points of intervention.

\subsection{Discern-Aware Input Embeddings}
Standard embeddings $E(t)$ capture only semantic statistics.  
We extend them with discernment weights:
\[
  E'(t) = [E(t), D(t)],
\]
where $D(t)\in[0,1]$ measures structural validity (syntax, logic, or knowledge consistency).
This ensures Discern is injected at the very first layer.

\subsection{Structural Layer Normalization}
LayerNorm currently stabilizes variance but ignores structure.
We propose:
\[
  h' = \frac{h - \mu}{\sigma} \cdot f(D),
\]
where $f(D)$ rescales activations based on Discern weights.
Low-discernment tokens thus contribute less to gradient flow.

\subsection{Discern-Gated Residual Connections}
Residuals propagate all information: $y = x + F(x)$.  
SAE modifies this as:
\[
  y = x + D \odot F(x),
\]
where $D$ gates contributions of each token or channel.
Invalid trajectories no longer accumulate noise.

\subsection{KV-Cache Pruning}
In long-context inference, the KV-cache grows linearly.
We restrict storage to discerned keys:
\[
  \text{KV}' = \{(k,v) \mid D(k) \geq \tau \}.
\]
Only structurally valid tokens remain in memory, improving efficiency.

\subsection{SAE-Decoding Strategies}
Decoding methods such as Top-$k$ or nucleus sampling consider only probabilities.
We redefine sampling distribution as:
\[
  p_i' = \frac{p_i D_i}{\sum_j p_j D_j}.
\]
This guarantees Discern is enforced during generation itself,
not only as a post-filter.

\subsection{Training Paradigm Beyond Cross-Entropy}
We extend the objective:
\[
  \mathcal{L} = \mathcal{L}_{\text{task}}
  + \alpha\,\mathcal{L}_{\text{sync}}
  + \beta\,\mathcal{L}_{\text{energy}}
  + \gamma\,\mathcal{L}_{\text{discern-calib}}.
\]
Discern calibration stabilizes $D$ estimators, while
synchronization and energy regularizers optimize trajectories as
``existence paths'' rather than mere token prediction.

\subsection{Cross-Modal Discernment}
For multimodal LLMs, Discern must be aligned across modalities:
\[
  D^{\text{multi}} = f(D^{\text{text}}, D^{\text{vision}}, D^{\text{audio}}).
\]
This prevents hallucinatory cross-modal associations and ensures
valid semantic alignment across input channels.

\subsection*{Summary}
SAE acts not as a small modification but as a global design principle:
embedding, normalization, residuals, memory, decoding, training,
and multimodal fusion can all be redefined under
$\mathsf{Discern}\wedge \mathsf{Free}$,
leading toward a ``second-generation Transformer'' architecture.

\end{document}